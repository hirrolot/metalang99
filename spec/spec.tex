\documentclass[12pt]{article}

\usepackage{hyperref}
\usepackage{float}
\usepackage{bussproofs}
\usepackage{minted}
\usepackage{amsmath}
\usepackage{amssymb}
\usepackage[standard,thmmarks]{ntheorem}
\usepackage{biblatex}
\usepackage{csquotes}
\usepackage{mathtools}
\usepackage{tabularx}
\usepackage[english]{babel}

\theoremstyle{break}
\newtheorem{notation}{Notation}

\addbibresource{references.bib}

\floatstyle{boxed}
\restylefloat{figure}
\allowdisplaybreaks

\begin{document}

\title{Metalang99 Specification}
\date{\today}
\author{Hirrolot \\ e-mail: \href{mailto:hirrolot@gmail.com}{hirrolot@gmail.com}}
\maketitle

\begin{abstract}
This paper formally describes the syntax and semantics of metaprograms written in Metalang99,
a metalanguage aimed at full-blown C99 preprocessor metaprogramming. For the motivation and
a user-friendly overview, please see the official repository \cite{Metalang99}.
\end{abstract}

\tableofcontents

\newpage

\section{EBNF Grammar}

\begin{figure}[H]
    \caption{Grammar rules}

\begin{minted}{bnf}
<eval> ::= "ML99_eval(" <term-seq> ")" ;

<term> ::=
      "ML99_call(" <op> "," <term-seq> ")"
    | "ML99_callTrivial(" <ident> "," <pp-token-list> ")"
    | "ML99_abort(" <term-seq> ")"
    | "ML99_fatal(" <ident> "," <pp-token-list> ")"
    | "v(" <pp-token-list> ")" ;

<op> ::= <ident> | <term> ;

<term-seq> ::= <term> { "," <term> }* ;
\end{minted}

\end{figure}

Notes:

\begin{itemize}
    \item \texttt{<pp-token-list>} stands for a list of preprocessor tokens (e.g., \texttt{a 123, hello!}).
    \item The grammar above describes metaprograms already expanded by the preprocessor,
    except for \texttt{ML99\_eval}, \texttt{ML99\_call}, \texttt{ML99\_callTrivial},
    \texttt{ML99\_abort}, \texttt{ML99\_fatal}, and \texttt{v}.
    \item \texttt{ML99\_call} accepts \texttt{op} either as an identifier or as a term that
    reduces to an identifier.
    \item \texttt{ML99\_callTrivial} accepts an operation strictly as an identifier.
\end{itemize}

The \texttt{ML99\_call} syntax hurts IDE support (no parameters documentation highlighting) and is also
less natural. The workaround is to define a wrapper around an implementation macro like this:

\begin{minted}{c}
/// The documentation string.
#define FOO(a, b, c) ML99_call(FOO, a, b, c)
#define FOO_IMPL(a, b, c) // The implementation.
\end{minted}

Then \texttt{FOO} can be conveniently called as \texttt{FOO(v(1), v(2), v(3))}.

\section{Notations}

\begin{notation}[Sequence]
    \begin{itemize}
        \item $\overline{x} \coloneqq x_1 \ldots x_n$. Examples:
        \begin{itemize}
            \item Metalang99 terms: \texttt{v(abc), ML99\_call(FOO, v(123)), v(u 8 9)}
            \item Preprocessor tokens: \texttt{abc 13 "hello" + -}
        \end{itemize}
        \item $()$ denotes the empty sequence.
        \item Appending to a sequence:
        \begin{itemize}
            \item Appending an element: $S \ y \coloneqq x_1 \ldots x_n \ y$, where $S = x_1 \ldots x_n$
            \item Appending a sequence: $S_1 \ S_2 \coloneqq x_1 \ldots x_n \ y_1 \ldots y_m$, where $S_1 = x_1 \ldots x_n$
            and $S_2 = y_1 \ldots y_m$
        \end{itemize}
    \end{itemize}
\end{notation}

\begin{notation}[Reduction step]
    $\to$ denotes a single step of reduction (computation, evaluation).
\end{notation}

\begin{notation}[Metavariables]
    \ \\
    \begin{tabular}{|c|c|}
        \hline
        \texttt{tok} & preprocessor token \\
        \texttt{ident} & preprocessor identifier \\
        \texttt{t} & Metalang99 term \\
        \texttt{a} & Metalang99 term used as an argument \\
        \hline
    \end{tabular}
\end{notation}

\section{Reduction Semantics}

We define a reduction semantics for Metalang99 \ref{ReductionSemantics}. The abstract
machine executes configurations of the form $\langle K; F; A; C \rangle$:

\begin{itemize}
    \item $K$ is a continuation of the form $\langle K; F; A; C \rangle$, where
    $C$ includes the $?$ sign denoting a result passed into the continuation.
    For example, let $K$ be $\langle K'; (1, 2, 3); v(x), \ ? \rangle$,
    then $K(v(y))$ is $\langle K'; (1, 2, 3); v(x), v(y) \rangle$. A special
    continuation $halt$ terminates the abstract machine and substitutes itself
    with a provided result. For example, when the abstract machine encounters
    $halt(1 + 2)$, it will just stop and paste $1 + 2$.

    \item $F$ is a left folder of the form $(acc, \overline{tok}) \to acc$. It is used
    to flexibly append a newly evaluated term to an accumulator without extra reduction
    steps. There are the only two folders:
    \begin{itemize}
        \item $fappend(acc, \overline{tok}) \coloneqq acc \ \overline{tok}$
        \item $fcomma(acc, \overline{tok}) \coloneqq if (acc \ is \ ()) \ then \ \overline{tok} \ else \ acc \ "," \ \overline{tok}$
    \end{itemize}

    \item $A$ (accumulator) is a sequence of already computed results.

    \item $C$ (control) is a sequence of terms upon which the abstract
    machine is operating right now.
\end{itemize}

\begin{figure}
    \caption{Reduction Semantics}

    \begin{align*}
        (v): & \ \langle K; F; A; \texttt{v}(\overline{tok}), \overline{t} \rangle \to
            \langle K; F; F(A, \overline{tok}); \overline{t} \rangle \\
        (op): & \ \langle K; F; A; \texttt{ML99\_call}(t, \overline{a}), \overline{t'} \rangle \to \langle \\
            & \ \ \ \ \langle K; F; A; \texttt{ML99\_call}(?), \overline{t'} \rangle; \\
            & \ \ \ \ fcomma; \\
            & \ \ \ \ (); \\
            & \ \ \ \ t, \overline{a} \rangle \\
        (args): & \ \langle K; F; A; \texttt{ML99\_call}(ident, \overline{a}), \overline{t} \rangle \to \langle \\
            & \ \ \ \ \langle \langle K; F; F(A, ?); \overline{t} \rangle; fappend; (); ident\texttt{\_IMPL}(?) \rangle; \\
            & \ \ \ \ fcomma; \\
            & \ \ \ \ (); \\
            & \ \ \ \ \overline{a} \rangle \\
        (callTrivial): & \ \langle K; F; A; \texttt{ML99\_callTrivial}(ident, \overline{tok}), \overline{t} \rangle \to \langle \\
            & \ \ \ \ \langle K; F; F(A, ?); \overline{t} \rangle; \\
            & \ \ \ \ fappend; \\
            & \ \ \ \ (); \\
            & \ \ \ \ ident\texttt{\_IMPL}(\overline{tok}) \rangle \\
        (abort): & \ \langle K; F; A; \texttt{ML99\_abort}(\overline{t}), \overline{t'} \rangle \to \langle halt; fappend; (); \overline{t} \rangle \\
        (fatal): & \ \langle K; F; A; \texttt{ML99\_fatal}(ident, \overline{tok}), \overline{t} \rangle \to halt(\ldots) \\
        (end): & \ \langle K; F; A; () \rangle \to K(A) \\
        (start): & \ \texttt{ML99\_call}(\overline{t}) \to \langle halt; fappend; (); \overline{t} \rangle
    \end{align*}
    \label{ReductionSemantics}
\end{figure}

Notes:

\begin{itemize}
    \item Metalang99 follows applicative evaluation strategy \cite{ApplicativeEvaluationStrategy}.

    \item $(args)$ Metalang99 generates a usual C-style macro invocation with
    fully evaluated arguments, which will be then expanded by the preprocessor, resulting
    in yet another concrete sequence of Metalang99 terms to be evaluated by the computational
    rules.
    \item $(args)$ Metalang99 appends \texttt{\_IMPL} to every macro identifier called using
    \texttt{ML99\_call} -- it makes easier to follow the convention that all implementations
    of metafunctions must have the postfix \texttt{\_IMPL}.
    \item $(callTrivial)$ \texttt{ML99\_callTrivial} is used when an operation and all
    arguments are already evaluated. It is semantically the same as \\ \texttt{ML99\_call(ident, v(...))}
    but performs one less reduction steps to benefit in performance.
    \item $(fatal)$ The ellipsis means that an implementation is free to provide
    diagnostics in any format.
    \item $(fatal)$ interprets its variadic arguments without preprocessor expansion -- i.e.,
    they are pasted as-is. This is intended because otherwise identifiers located in an
    error message may stand for other macros that will be unintentionally expanded.
    \item With the current implementation, at most $2^{14}$ reduction steps are
    possible. After exceeding this limit, Metalang99 will not be able to perform reduction
    of a given metaprogram anymore.
\end{itemize}

\subsection{Examples}

Take the following code:

\begin{minted}{c}
#define X_IMPL(op)        ML99_call(op, v(123))
#define CALL_X_IMPL(_123) ML99_call(X, v(ID))
#define ID_IMPL(x)        v(x)
\end{minted}

See how \texttt{ML99\_call(X, v(CALL\_X))} is evaluated:

\begin{example}[Evaluation of terms]
\begin{align*}
    \texttt{ML99\_eval}(\texttt{ML99\_call}(X, \texttt{v}(\texttt{CALL\_X}))) & \ \underrightarrow{(start)} \\
    \langle halt; fappend; (); \texttt{ML99\_call}(X, \texttt{v}(\texttt{CALL\_X})) \rangle & \ \underrightarrow{(args)} \\
    \langle \langle halt; fappend; (); \texttt{X}(?) \rangle; fcomma; (); \texttt{v}(\texttt{CALL\_X}) \rangle & \ \underrightarrow{(v)} \\
    \langle \langle halt; fappend; (); \texttt{X}(?) \rangle; fcomma; \texttt{CALL\_X}; () \rangle & \ \underrightarrow{(end)} \\
    \langle halt; fappend; (); \texttt{ML99\_call}(\texttt{CALL\_X}, \texttt{v}(123)) \rangle & \ \underrightarrow{(args)} \\
    \langle \langle halt; fappend; (); \texttt{CALL\_X}(?) \rangle; fcomma; (); \texttt{v}(123) \rangle & \ \underrightarrow{(v)} \\
    \langle \langle halt; fappend; (); \texttt{CALL\_X}(?) \rangle; fcomma; 123; () \rangle & \ \underrightarrow{(end)} \\
    \langle halt; fappend; (); \texttt{ML99\_call}(X, \texttt{v}(\texttt{ID})) \rangle & \ \underrightarrow{(args)} \\
    \langle \langle halt; fappend; (); \texttt{X}(?) \rangle; fcomma; (); \texttt{v}(\texttt{ID}) \rangle & \ \underrightarrow{(v)} \\
    \langle \langle halt; fappend; (); \texttt{X}(?) \rangle; fcomma; \texttt{ID}; \rangle & \ \underrightarrow{(end)} \\
    \langle halt; fappend; (); \texttt{ML99\_call}(\texttt{ID}, \texttt{v}(123)) \rangle & \ \underrightarrow{(args)} \\
    \langle \langle halt; fappend; (); \texttt{ID}(?) \rangle; fcomma; (); \texttt{v}(123) \rangle & \ \underrightarrow{(v)} \\
    \langle \langle halt; fappend; (); \texttt{ID}(?) \rangle; fcomma; 123; () \rangle & \ \underrightarrow{(end)} \\
    \langle halt; fappend; (); \texttt{v}(123) \rangle & \ \underrightarrow{(v)} \\
    \langle halt; fappend; 123; () \rangle & \ \underrightarrow{(end)} \\
    halt(123) &
\end{align*}
\end{example}

The analogous version written in ordinary C looks like this:

\begin{minted}{c}
#define X(op)        op(123)
#define CALL_X(_123) X(ID)
#define ID(x)        x
\end{minted}

However, unlike the Metalang99 version above, \texttt{X(CALL\_X)} gets blocked \cite{Blueprinting} due to the
second call to \texttt{X}. The trick is that Metalang99 performs evaluation step-by-step,
unlike the preprocessor:

\begin{itemize}
    \item The Metalang99 version: \texttt{X(CALL\_X)} expands to \texttt{ML99\_call(\\ CALL\_X, v(123))}.
    This expansion does not contain \texttt{X}, and therefore \texttt{X} is \textbf{not}
    blocked by the preprocessor.

    \item The ordinary version: \texttt{X(CALL\_X)} expands to \texttt{X(ID)}. This expansion
    does contains \texttt{X}, and therefore \texttt{X} is blocked by the preprocessor.
\end{itemize}

\section{Properties}

\subsection{Progress}

\begin{proposition}[Progress]
Either $\langle K; F; A; \overline{t} \rangle \to \langle K; F; A; \overline{t'} \rangle$ or \\
$\langle K; F; A; \overline{t} \rangle \to halt(\overline{x})$.
\end{proposition}

\begin{proof}
By inspection of \ref{ReductionSemantics}.
\end{proof}

\section{Caveats}

\begin{itemize}
\item Consider this scenario:
    \begin{itemize}
        \item You call \texttt{FOO(1, 2, 3)}
        \item It gets expanded by the preprocessor (not by Metalang99)
        \item Its expansion contains \texttt{FOO}
    \end{itemize}
Then \texttt{FOO} gets blocked \cite{Blueprinting} by the preprocessor, i.e. Metalang99 cannot handle ordinary
macro recursion; you must use \texttt{ML99\_call} to be sure that recursive calls
will behave as expected. I therefore recommend to use only primitive C-style macros, e.g.
for performance reasons or because of you cannot express them in terms of Metalang99.
\end{itemize}

\emergencystretch=1em
\printbibliography

\end{document}
